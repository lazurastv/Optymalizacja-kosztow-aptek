\documentclass{article}

\usepackage{polski}

\title{Wprowadzone zmiany}
\date{29.11.2020}
\author{Dominik Wawrzyniuk}

\begin{document}

\maketitle

\section{Zmiany}

\begin{enumerate}
	\item Zamiast tworzyć oddzielne kontenery na budynki i handle, powstał kontener wskaźników na ich tablice
	\item Sortowanie jest wykonywane po przeczytaniu wszystkich danych
	\item Czytnik pliku został zapisany we własnej klasie
	\item Pozbyto się plików conv, przenosząc ich funkcjonalność na konstruktor grafu
	\item Zamieniono typ float w kosztach na typ double
	\item Struktura wynik nie przechowuje sumy
	\item Algorytm został zapisany we własnej klasie, pozbyto się funkcji aktualizujWynik
	\item Zmieniono strukturę katalogów
	\item Zamiast zapisywać plik do ustalonego folderu, program sam ten folder tworzy
	\item Nagłówki w pliku wejściowym zostały ograniczone do nagłówków bez polskich znaków
\end{enumerate}

\newpage{}

\section{Powody zmian}

\begin{enumerate}
	\item Jest to skuteczniejsze rozwiązanie i unika tworzenia dwóch bardzo podobnych klas
	\item Sortowanie podczas wstawiania jest wolniejsze niż posortowanie wszystkich wartości naraz
	\item Pozwala to na wewnętrzne przechowywanie różnych używanych zmiennych, m. in. czytanej linijki lub pliku
	\item Jest to prostsze rozwiązanie, a im mniej plików, tym lepiej
	\item Więcej liczb o precyzji dwóch miejsc po przecinku można przedstawić dokładnie typem double niż typem float
	\item Nie ma potrzeby żeby przechowywała, skuteczniej jest obliczyć sumę kosztów zakupów
	\item Pozwala to na przechowywanie stanu rozwiązania w klasie, umożliwiając wygodną modyfikację rozwiązania
	\item Poprawia ona czytelność i jest dostosowana do standardowych struktur katalogowych języków C/C++
	\item Pozwala to na uruchamianie programu gdziekolwiek, nie tylko w folderze projektu
	\item Chroni to przed błędnym porównywaniem polskich znaków z oczekiwanymi
\end{enumerate}

\end{document}