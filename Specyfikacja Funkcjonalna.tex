\documentclass{article}

\usepackage{polski}
\usepackage{dirtree}

\title{Specyfikacja Funkcjonalna}
\date{07.11.2020}
\author{Dominik Wawrzyniuk}

\begin{document}

\maketitle

\section{Opis ogólny}

Nazwa programu to OLKA, co jest skrótem od Optymalizacja Leków i Kosztów Aptek. Zadaniem tego programu jest znalezienie konfiguracji zakupów u różnych dostawców takiej, że łączny koszt jest najmniejszy, a zapotrzebowanie wszystkich aptek jest spełnione. Program dedykowany jest dla grupy menadżerskiej sieci aptek.

\section{Opis funkcjonalności}

Program uruchamia się poprzez plik .exe. Należy wywołać go, podając jako parametr ścieżkę do pliku wejściowego.

\paragraph{}

Program będzie potrafił:

\begin{itemize}
	\item Odczytać dane z podanego pliku
	\item Wykryć błędy w pliku wejściowym
	\item Znaleźć optymalną konfigurację zakupów
	\item Zapisać znalezioną konfigurację do pliku wynikowego
\end{itemize}

\section{Format danych i struktura plików}

Główny folder projektu będzie nazywać się OLKA. 
Wszystkie pliki związane z programem znajdują się w tym folderze. Będzie on zawierał trzy foldery: folder z zaimplementowanym algorytmem, folder z modułem sterującym programem oraz folder z plikiem wyjściowym. W folderze OLKA będzie znajdować się plik .exe.
Na następnej stronie znajduje się model struktury katalogów.

\newpage

\dirtree{%
	.1 OLKA.
	.2 Algorytm.
	.2 Moduł główny.
	.2 Plik wyjściowy.
}

\paragraph{}

Wymaganym plikiem wejściowym jest plik tekstowy zawierający informacje dotyczące:

\begin{itemize}
	\item maksymalnej produkcji leków przez dostawców
	\item zapotrzebowania leków przez apteki
	\item umowy między poszczególnymi dostawcami i aptekami (maksymalna dostawa, cena za sztukę)
\end{itemize}

Dane te muszą być podane dokładnie wg. następującego schematu: \newline

\noindent
\#  Producenci szczepionek (id $|$ nazwa $|$ dzienna produkcja) \newline
0 $|$ BioTech 2.0 $|$ 900 \newline
... \newline
\# Apteki (id $|$ nazwa $|$ dzienne zapotrzebowanie) \newline
0 $|$ CentMedEko Centrala $|$ 450 \newline
... \newline
\# Połączenia producentów i aptek (id producenta $|$ id apteki $|$ dzienna maksymalna liczba dostarczanych szczepionek $|$ koszt szczepionki [zł] ) \newline
0 $|$ 0 $|$ 800 $|$ 70.5 \newline
...

\paragraph{}

Pliki nie dostosowane do tego formatu nie zostaną przyjęte.

\paragraph{}

Program będzie przechowywać dane wczytane z pliku w tablicach, wskaźnikach i macierzach.

\paragraph{}

Jako plik wyjściowy program zwróci plik tekstowy zawierający informacje dotyczące ile dana apteka ma kupić u danego dostawcy. Jeżeli apteka nie kupuje u danego dostawcy, handel ten nie będzie wymieniony w pliku.

\section{Scenariusz działania programu}

Scenariusz z punktu widzenia użytkownika:

\begin{enumerate}
	\item Uruchamiamy program plikiem .exe., podając jako parametr ścieżkę do naszego pliku wejściowego.
	\item Jeżeli wystąpi błąd, program go zgłosi i zakończy działanie. Należy błąd naprawić żeby program poprawnie zadziałał.
	\item Program umieści plik wyjściowy w przeznaczonym do tego folderze.
\end{enumerate}

\newpage

\noindent Scenariusz z punktu widzenia programu:

\begin{enumerate}
	\item Program zostaje wywołany.
	\item Moduł główny sprawdza wystąpienie któregoś z następujących błędów:
	\begin{enumerate}
		\item Brak pliku wejściowego.
		\item Niepoprawny rodzaj pliku wejściowego (nie plik tekstowy).
		\item Nieprzyjmowany format zapisu danych w pliku.
		\item Niekompletny plik wejściowy (brak niektórych wartości).
		\item Niepoprawne dane w pliku (liczby ujemne, niecałkowite dostawy leków).
		\item Sprzeczności danych w pliku (odwołanie do dostawcy/apteki która nie istnieje).
	\end{enumerate}
	\item Jeżeli wystąpił błąd, program go zgłasza i kończy działanie.
	\item Moduł główny zapisuje dane z pliku wejściowego w macierzy przyjmowanej przez algorytm.
	\item Moduł główny przekazuje dane algorytmowi.
	\item Algorytm oblicza optymalne rozwiązanie dla danych wejściowych i przekazuje je modułowi.
	\item Moduł zapisuje dane w pliku wyjściowym i program kończy swoje działanie.
\end{enumerate}

Program nie posiada graficznego interfejsu użytkownika (GUI). 

\section{Testowanie}

Moduł główny zostanie przetestowany pod względem wszystkich błędów, które mogłyby wystąpić podczas wczytywania danych wejściowych, oraz pod względem poprawnego przekazywania danych. Algorytm zostanie przetestowany na kilku przykładach, a optymalność jego rozwiązań zostanie porównana z optymalnym rozwiązaniem znalezionym przeze mnie.

\end{document}